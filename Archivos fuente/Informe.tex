\documentclass[a4paper, 12pt]{article}
\usepackage[spanish]{babel}
\usepackage[utf8]{inputenc}
\usepackage{amsmath}
\usepackage{listings}
\usepackage{xcolor}
\usepackage{graphicx}
\usepackage{hyperref}
\usepackage{geometry}
\usepackage{array}
\usepackage{booktabs}
\geometry{margin=2.5cm}

\definecolor{codegreen}{rgb}{0,0.6,0}
\definecolor{codegray}{rgb}{0.5,0.5,0.5}
\definecolor{codepurple}{rgb}{0.58,0,0.82}
\definecolor{backcolour}{rgb}{0.95,0.95,0.92}

\lstdefinestyle{mips}{
    backgroundcolor=\color{backcolour},
    commentstyle=\color{codegreen},
    keywordstyle=\color{magenta},
    numberstyle=\tiny\color{codegray},
    stringstyle=\color{codepurple},
    basicstyle=\ttfamily\footnotesize,
    breakatwhitespace=false,
    breaklines=true,
    captionpos=b,
    keepspaces=true,
    numbers=left,
    numbersep=5pt,
    showspaces=false,
    showstringspaces=false,
    showtabs=false,
    tabsize=4,
    frame=single
}

\title{Práctica de Laboratorio 1: Secuencia de Fibonacci en MIPS32}
\author{Nombre del Estudiante}
\date{Julio 2025}

\begin{document}

\maketitle

\section{Implementación de recursividad en MIPS32}
\begin{itemize}
    \item \textbf{Mecanismo:} Llamadas anidadas mediante \texttt{jal} con gestión manual de contexto
    \item \textbf{Rol de la pila (\texttt{\$sp}):}
    \begin{itemize}
        \item Almacena dirección de retorno (\texttt{\$ra})
        \item Preserva argumentos (ej. \texttt{\$a0} con valor actual de \texttt{n})
        \item Guarda resultados intermedios
        \item Maneja crecimiento dinámico: \texttt{addi \$sp, \$sp, -12} (reserva) y \texttt{addi \$sp, \$sp, 12} (liberación)
    \end{itemize}
\end{itemize}

\section{Riesgos de desbordamiento y mitigación}
\begin{table}[h]
    \centering
    \begin{tabular}{|l|l|l|}
        \hline
        \textbf{Riesgo} & \textbf{Causa} & \textbf{Mitigación} \\
        \hline
        Stack overflow & Recursión profunda (\texttt{n} > 1000) & Limitar \texttt{n} ≤ 40 \\
        \hline
        Integer overflow & \texttt{fib(47)} > $2^{31}-1$ & Validar \texttt{n} ≤ 46 \\
        \hline
        Heap overflow & Memoria dinámica con \texttt{n} grande & Liberar memoria con \texttt{sbrk} negativo \\
        \hline
    \end{tabular}
\end{table}

\section{Comparación: Iterativo vs. Recursivo}
\begin{table}[h]
    \centering
    \begin{tabular}{|l|c|c|}
        \hline
        \textbf{Característica} & \textbf{Iterativo} & \textbf{Recursivo} \\
        \hline
        Complejidad temporal & $O(n)$ & $O(2^n)$ \\
        \hline
        Uso memoria & Heap: $4(n+1)$ bytes & Pila: $12n$ bytes \\
        \hline
        Registros usados & Temporales (\texttt{\$tX}) & Preservados (\texttt{\$sX}) \\
        \hline
        Legibilidad & Más extenso & Similar a definición matemática \\
        \hline
    \end{tabular}
\end{table}

\section{Diferencias con ejemplos académicos}
\begin{itemize}
    \item \textbf{Código real incluye:}
    \begin{itemize}
        \item Validación de entrada (\texttt{n} ≥ 0)
        \item Gestión de memoria dinámica (\texttt{sbrk})
        \item Mensajes de error específicos
        \item Interfaz de usuario completa (prompts, formatos)
        \item Visualización de secuencia completa (iterativo)
    \end{itemize}
    \item \textbf{Ejemplos académicos:} Focalizados en algoritmo básico sin manejo de errores
\end{itemize}

\section{Tutorial de Ejecución Paso a Paso en MARS}
\subsection{Preparación Inicial}
\begin{enumerate}
    \item Descargar e instalar MARS (versión 4.5 o superior)
    \item Configurar:
    \begin{itemize}
        \item \texttt{Settings → Memory Configuration → Compact, Data at Address 0}
        \item \texttt{Settings → Delayed Branches → Disabled}
    \end{itemize}
\end{enumerate}

\subsection{Ejecución del Fibonacci Iterativo (n=5)}
\begin{enumerate}
    \item \textbf{Cargar y ensamblar:}
        \begin{itemize}
            \item Abrir \texttt{Fibonacci\_Iterativo.asm}
            \item Ensamblar (F3)
        \end{itemize}
    
    \item \textbf{Flujo de ejecución:}
        \begin{enumerate}
            \item \texttt{main}: Muestra prompt y lee entrada (ingresar 5)
            \item Reserva de memoria:
                \begin{itemize}
                    \item \texttt{addi \$t0, \$s1, 1}: \$t0 = 6 (n+1)
                    \item \texttt{sll \$a0, \$t0, 2}: \$a0 = 24 bytes
                    \item \texttt{li \$v0, 9}: syscall sbrk → \$s2 = dirección base (ej: 0x10040000)
                \end{itemize}
            \item \texttt{fib\_array}:
                \begin{itemize}
                    \item \texttt{sw \$zero, 0(\$s2)}: fib[0] = 0
                    \item \texttt{li \$t0, 1; sw \$t0, 4(\$s2)}: fib[1] = 1
                    \item Bucle (i=2 a 5):
                        \begin{table}[h]
                            \centering
                            \begin{tabular}{|c|c|c|}
                                \hline
                                \textbf{i} & \textbf{Dirección} & \textbf{Valor} \\
                                \hline
                                2 & 0x10040008 & 1 (0+1) \\
                                \hline
                                3 & 0x1004000C & 2 (1+1) \\
                                \hline
                                4 & 0x10040010 & 3 (1+2) \\
                                \hline
                                5 & 0x10040014 & 5 (2+3) \\
                                \hline
                            \end{tabular}
                        \end{table}
                \end{itemize}
            \item Salida:
                \begin{itemize}
                    \item Muestra: "El número Fibonacci F(5) es: 5"
                    \item Secuencia completa: "0 1 1 2 3 5"
                \end{itemize}
        \end{enumerate}
\end{enumerate}

\subsection{Ejecución del Fibonacci Recursivo (n=3)}
\begin{enumerate}
    \item \textbf{Cargar y ensamblar:} \texttt{Fibonacci\_Recursivo.asm} (F3)
    \item \textbf{Flujo de ejecución:}
        \begin{enumerate}
            \item \texttt{main}: Lee n=3, llama a \texttt{fib\_rec}
            \item \textbf{Pila durante ejecución:}
                \begin{table}[h]
                    \centering
                    \begin{tabular}{|c|c|c|c|}
                        \hline
                        \textbf{Llamada} & \texttt{\$ra} & \texttt{\$a0} & \textbf{Resultado} \\
                        \hline
                        fib\_rec(3) & 0x00400034 & 3 & ? \\
                        \hline
                        fib\_rec(2) & 0x004000A8 & 2 & ? \\
                        \hline
                        fib\_rec(1) & 0x004000A8 & 1 & 1 \\
                        \hline
                        fib\_rec(0) & - & 0 & 0 \\
                        \hline
                    \end{tabular}
                \end{table}
            \item \textbf{Desarrollo:}
                \begin{itemize}
                    \item \texttt{fib\_rec(3)}: Guarda \$ra=0x00400034, \$a0=3. Llama a fib\_rec(2).
                    \item \texttt{fib\_rec(2)}: Guarda \$ra=0x004000A8, \$a0=2. Llama a fib\_rec(1) y fib\_rec(0).
                    \item \texttt{fib\_rec(1)}: Retorna 1 → guardado en pila de fib\_rec(2).
                    \item \texttt{fib\_rec(0)}: Retorna 0.
                    \item \texttt{fib\_rec(2)}: Calcula 1+0=1 → retorna 1.
                    \item \texttt{fib\_rec(3)}: Recupera resultado fib(2)=1, calcula fib(1)=1 → 1+1=2 → retorna 2.
                \end{itemize}
            \item \texttt{main}: Recibe 2, muestra "F(3) es: 2"
        \end{enumerate}
\end{enumerate}

\section{Justificación de enfoque}
\textbf{Iterativo recomendado cuando:}
\begin{itemize}
    \item $n > 30$ (evita stack overflow)
    \item Eficiencia crítica ($O(n)$ vs $O(2^n)$)
    \item Memoria limitada
\end{itemize}

\textbf{Recursivo recomendado cuando:}
\begin{itemize}
    \item $n \leq 30$ y claridad > rendimiento
    \item Propósitos educativos
\end{itemize}

\textbf{Conclusión:} Para MIPS32 el enfoque iterativo es superior en estabilidad y rendimiento.

\section{Análisis de resultados}
\begin{itemize}
    \item \textbf{Límites prácticos:}
    \begin{itemize}
        \item Recursivo: $n=40$ tarda >10s
        \item Iterativo: $n=10^6$ en <1s (teórico)
    \end{itemize}
    \item \textbf{Consumo memoria:}
    \begin{itemize}
        \item Recursivo: $\approx$ 12n bytes (pila)
        \item Iterativo: $4(n+1)$ bytes (heap)
    \end{itemize}
    \item \textbf{Optimizaciones identificadas:}
    \begin{itemize}
        \item Iterativo: Almacenar sólo últimos 2 valores (reduce memoria a O(1))
        \item Recursivo: Memoization (caché de resultados)
    \end{itemize}
\end{itemize}

\section*{Conclusiones}
La implementación iterativa es preferible para aplicaciones reales en MIPS32 por su eficiencia y estabilidad, mientras la recursiva sirve como herramienta educativa para demostrar el manejo de pila en llamadas anidadas.

\end{document}